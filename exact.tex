\section{A polynomial algorithm for\\perfect role assignment}
\label{section:perfect}


Before presenting our proposed algorithms for the \prbrm problem, 
we first present a polynomial algorithm
for finding a perfect role assignment --- a solution with cost zero. 
In the light of the previous \np-hardness results, 
this polynomial algorithm appears somewhat surprising. 
Note however, that this polynomial algorithm 
guarantees to find the minimum number of roles for which there is a solution of cost zero. 
This is a different question than the one posed by the \prbrm problem, 
where we aim to find a minimum-cost solution for a given number of roles.

The polynomial algorithm, named \algperfect, 
works by first setting $k=1$ and assigning all vertices to the same role. 
Then, it iteratively computes the profile of each vertex, 
and groups together all vertices with the same profile. 
For each new group it then assigns a new role (and increases $k$), 
and the iterative process continues as long as new roles are created. 
Pseudocode for \algperfect is given in Algorithm~\ref{alg:perfect}.

\begin{algorithm}[t]
\caption{$\algperfect(G)$, computes a perfect assignment with smallest number of roles.}
\label{alg:perfect}
	$\role(v) \define 1$ for every $v \in V$\;
	\While {number of roles increases} {
		compute profiles $\prof{v; \role}$\;
		group vertices with the same profiles\;
		assign a role to each group\;
	}
\end{algorithm}

Note that in the first iteration of \algperfect
the profile of each vertex $v$ is a scalar (1-dimensional vector) 
equal to the degree of the vertex.
Thus, during this first iteration
all vertices with the same degree will be grouped together and will be assigned the same role. 
In subsequent iterations the vertices with the same degree
will be potentially further subdivided to smaller groups, 
and vertices within a group are assigned to the same role. 

As the number of roles can only increase during the execution of \algperfect, 
the previous observation implies that \algperfect 
always returns a solution in which the number of roles $k$ is
at least as large as the number of distinct degrees in the graph. 
To verify this property, 
notice that it is not possible to obtain a perfect role assignment
with a smaller number of roles than the number of distinct degrees. 

Finally, note that the \algperfect algorithm always terminates, 
in the worst case when $k=n$ and each vertex is assigned to a unique role.

The main claim characterizing the optimality of the \algperfect algorithm
is formulated in the following proposition.

\begin{proposition}
\label{prop:perfectcorrect}
Algorithm~\ref{alg:perfect} converges in $\bigO{n}$ iterations. 
The role assignment returned by the algorithm  
provides a zero-cost solution, 
and has the smallest possible number of roles.
\end{proposition}

To prove the proposition, we will introduce a \emph{partial order} 
relationship between 
role-assignment functions:  
Given two role assignments $\role$ and $\role'$, we say that $\role$ is a
\emph{refinement} of~$\role'$ if $\role(v) = \role(w)$ implies $\role'(v) =
\role'(w)$. We will denote this relationship by $\role \preceq \role'$.

The following lemma describes the relationship between the profiles of refined role assignments.

\begin{lemma}
\label{lem:refineprofile}
Let $\role$ and $\role'$ be two role assignments such that $\role \preceq \role'$.
Consider two vertices % $v$ and $w$ 
with $\prof{v, \role} = \prof{w, \role}$. 
Then $\prof{v; \role'} = \prof{w; \role'}$.
\end{lemma}

\begin{IEEEproof}
Let $A = \set{\role(u); u \in V}$ and $A' = \set{\role'(u); u \in V}$ be the
possible role values in the two role assignments. 
Let $i \in A$, and define $V(i) = \set{u \in V; \role(u) = i}$.
Since $\role \preceq \role'$, $\role'(u)$ is constant for any $u \in V(i)$.
Consequently, we can define a mapping 
$\funcdef{\eta}{A}{A'}$ 
from roles of $\role$
to roles of $\role'$ via the function
$\eta(i) = \role'(u)$, where $u$ is any vertex in $V(i)$.

By definition, $\eta(\role(u)) = \role'(u)$ for any $u \in V$. 
Consequently,
for any $j \in A'$ and any $u \in V$, 
the $j$-th coordinate of the profile vector $\prof{u; \role'}$
of $u$ in $\role'$ can be computed as  
\[
\prof{u; \role'}_j = \sum_{\ell\mid\eta(\ell) = j} \prof{u; \role}_\ell.
\]
This guarantees that if $\prof{v, \role} = \prof{w, \role}$, then $\prof{v; \role'} = \prof{w; \role'}$.
\end{IEEEproof}

Lemma~\ref{lem:refineprofile} implies 
that if a perfect role assignment $\role^*$
is a refinement of another assignment $\role$, 
then $\role$ assigns the same profile to all vertices
with the same role in the perfect role assignment  $\role^*$.

\begin{corollary}
\label{cor:refineprofile}
Let $\role^*$ be a perfect (zero-cost) role assignment.
Let $\role$ be a role assignment with $\role^* \preceq \role$.
Consider two vertices with $\role^*(v) = \role^*(w)$. 
Then $\prof{v; \role} = \prof{w; \role}$.
\end{corollary}

\begin{proof}
Since $\role^*(v) = \role^*(w)$ and 
since $\role^*$ does not incur any cost, 
we must have $\prof{v, \role^*} = \prof{w, \role^*}$.
Since $\role^* \preceq \role$,
Lemma~\ref{lem:refineprofile} proves the claim.
\end{proof}

Let us now consider the {\algperfect} algorithm
and denote by $\role_i$ the role assignment 
at the beginning of the $i$-th iteration of the algorithm.
It can be shown that a perfect role assignment  $\role^*$ 
is a refinement of the role assignments $\role_i$
generated by \algperfect.

\begin{lemma}
\label{lem:optimalrefine}
Let $\role_i$ be role assignment 
at the beginning of the $i$-th iteration \algperfect, and
let $\role^*$ be a perfect (zero-cost) role assignment. 
Then $\role^* \preceq \role_i$.
\end{lemma}


\begin{proof}
We will prove the lemma by induction. 
It obviously holds for $i = 1$.
Let $v$ and $w$ be two vertices with $\role^*(v) = \role^*(w)$.
By the inductive hypothesis we have  $\role^* \preceq \role_i$,
for which
Corollary~\ref{cor:refineprofile} implies that $\prof{v; \role_i} = \prof{w; \role_i}$.
Then, by definition of \algperfect
we get $\role_{i + 1}(v) = \role_{i + 1}(w)$, 
which implies $\role^* \preceq \role_{i+1}$ 
and completes the induction.
\end{proof}

It is also not hard to see that 
\algperfect generates role ass\-ign\-ments 
that are iteratively refining one another.

\begin{lemma}
\label{lem:progress}
$\role_{i + 1} \preceq \role_i$.
\end{lemma}

\begin{proof}
We will prove this result by induction, too. 
The base case $i = 1$ is obvious.
To prove the general case for $i > 1$
consider two vertices $v$ and $w$ with $\role_{i + 1}(w) = \role_{i + 1}(v)$.
By the definition of the algorithm, 
this can only happen if $\prof{v, \role_i} = \prof{w, \role_i}$.
By the induction assumption,
we have $\role_i \preceq \role_{i - 1}$, and Lemma~\ref{lem:refineprofile}
guarantees that $\prof{v, \role_{i  - 1}} = \prof{w, \role_{i  - 1}}$. 
By the definition of the algorithm, 
we have $\role_{i}(w) = \role_{i}(v)$, which proves the lemma.
\end{proof}

Since  \algperfect  produces role assignments 
that iteratively refine one another,
and since a perfect role assignment 
is always a refinement of all the intermediate assignments,
it follows that \algperfect will eventually 
find a perfect role assignment. 
To complete the proof of Proposition~\ref{prop:perfectcorrect}
and obtain the main claim in this Section, 
we also need to show that the solution 
returned by \algperfect has the smallest possible number of roles.
This is shown next.

\begin{proof}[of Proposition~\ref{prop:perfectcorrect}]
Let $\role^*$ be the role assignment with the zero cost and the smallest
number of distinct roles.
%Lemma~\ref{lem:optimalrefine} guarantees that $\role^* \preceq \role_i$ for any $i$.

Lemma~\ref{lem:progress} states that $\role_{i + 1} \preceq \role_i$. 
If $\role_i \neq \role_{i + 1}$, then the number of roles must increase. 
% (This guarantees that we can have at most $\bigO{n}$ iterations.)
By the termination condition of the algorithm
and the previous observation, 
when the algorithm terminates we have $\role_{i} = \role_{i + 1}$. 
This can only happen when
$\prof{v; \role_i} = \prof{w; \role_i}$ for any $v$ and $w$ with $\role_i(v) = \role_i(w)$. 
Consequently, $\role_i$ yields a zero-cost solution, so it is perfect role assignment. 
Lemma~\ref{lem:optimalrefine} guarantees that $\role^* \preceq \role_i$, 
which implies that $\role_i$ has at most as many roles as $\role^*$, 
and thus, $\role_i$ has the smallest number of distinct roles.
\end{proof}

Finally, we analyze the running time of \algperfect.

\begin{proposition}
The computational complexity of \algperfect is $\bigO{mn \log n}$.
\label{prop:exacttime}
\end{proposition}

\begin{proof}
At each iterations we increase the number of roles. Since the number of roles
is at most $n$, we can have at most $\bigO{n}$ iterations.
Next we analyze the running time of each iteration.

Let us represent the profiles with sorted sparse lists. The size of a profile
of a vertex $v$ is $\bigO{\dg{v}}$. The total time of constructing the profiles is
then
\[
	\bigO{\sum_{v \in V} \dg{v} \log \dg{v}} \subseteq \bigO{m \log n}.
\]

Grouping based on profiles can be done by sorting the profiles in
lexicographical order, and then comparing the consecutive profiles. 

We next analyze the time needed to sort the profiles in lexicographical order.
Notice that comparing the profiles  $\prof{v}$ and $\prof{w}$ of two vertices $v$ and $w$ 
cannot be done in constant time; 
instead time $\bigO{\min\{\dg{v}, \dg{w}\}}$ is required.

Assume that merge sort is used for ordering the profiles lexicographically.
Consider a counter $\Delta(v)$ defined as follows:
initially, $\Delta(v)$ is set to $0$.
Consider, that we are merging two sorted vertex lists during the merge sort.
Assume that we are comparing vertex $v$ from the first list and vertex $w$ from the second list,
and assume that we determine $\prof{v; \role} \leq \prof{w; \role}$.
In such case, we increase $\Delta(v)$ by $\dg{v}$.
If $\prof{v; \role} > \prof{w; \role}$, then we increase $\Delta(w)$ by $\dg{w}$.

The increase of $\Delta(v)$ or $\Delta(w)$ is an upper bound for the
time that is required for each comparison, so
it immediately follows that sorting can be done in time $\bigO{\sum_u \Delta(u)}$. 

Note whenever we increase the counter, we always advance to the next vertex in
the list. That is,
each counter $\Delta(v)$ can be increased only once during a single merge, 
and each vertex participates only in $\bigO{\log n}$ merges.
Consequently, $\Delta(v) \in \bigO{\dg{v} \log n}$. 
This implies that sorting the profiles using  merge sort requires time 
$\bigO{\sum_v \dg{v} \log n} = \bigO{m \log n}$.
\end{proof}
