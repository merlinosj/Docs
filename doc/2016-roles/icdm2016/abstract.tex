
We provide a new formulation for the 
problem of role discovery in graphs. 
Our definition is structural and recursive: 
two vertices should be assigned to the same role
if the roles of their neighbors, when viewed as multi-sets, are similar enough.
An attractive characteristic of our approach 
is that it is based on optimizing a well-defined objective function, 
and thus, contrary to previous approaches, 
the role-discovery task can be studied with the tools of combinatorial optimization.

We demonstrate that, when fixing the number of roles to be used, 
the proposed role-discovery problem is \np-hard, 
while another (seemingly easier) version of the problem is \np-hard to approximate.
On the positive side, 
despite the recursive nature of our objective function, 
we can show that finding a \emph{perfect} (zero-cost) role assignment
with the minimum number of roles can be solved in polynomial time. 
We do this by connecting the zero-cost role assignment with the notion of equitable partition.
% Unfortunately, this polynomially-time solvable version of the problem
% is not very useful in practice, as the minimum number of roles found tends to be large.
For the more practical version of the problem with fixed number of roles
we present two natural heuristic methods, 
and discuss how to make them scalable in large graphs.
