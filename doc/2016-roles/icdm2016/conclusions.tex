\section{Concluding remarks}
\label{sec:conclusions}

In this paper we propose a new type of role discovery optimization problem:
the vertices should have the same role if their profiles, role
counts of neighbors, are similar.

From technical point, our method is different than feature-based techniques
because our features are in fact roles of neighbors.
%Hence, a two-step
%approach---($i$) construct features and ($ii$) cluster roles from
%features---does not work.
This dependency makes the optimization
problem difficult: we show that the problem is \np-hard, and cannot be even approximated if we fix the
centroids for the roles.

On the positive side we show that we can discover the perfect, zero-cost,
solution with minimal number of roles efficiently in polynomial time. When the
number of roles is fixed, we propose two simple natural heuristics: iterative
optimization and a hill-climbing algorithm.

Interestingly enough, we do not directly use any network-based feature when
comparing vertices. Instead, we are only interested in role counts. Our logic
is that fundamentally different ego-networks for vertices, say, $u$ and $v$,
should result in different role counts which should imply that $u$ and $v$ are
different. Nevertheless, combining our approach with other feature-based role
discovery methods provides a potentially fruitful direction for future work. 
