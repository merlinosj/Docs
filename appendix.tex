\begin{proposition}
\label{prop:fastgain}
Assume a role assignment $\role$ with optimal centroids $\centroid$.
Let $v$ be a vertex and $j$ be an integer.
Let $i = \role(v)$.
Define $n_\ell = \abs{\set{u \in V ; \role(u) = \ell}}$ to be the number of vertices having role $\ell$.
Define $d_\ell = \abs{\set{(v, w) \in E ; \role(w) = \ell}}$ to be the number of neighbors of $v$ having role $\ell$.

Define a new role assignment $\role'$ obtained from $\role$ by setting $\role'(v) = j$.
Let $n'_\ell = \abs{\set{u \in V ; \role'(u) = \ell}}$.
Let $\centroid'$ be the optimal centroids with respect to $\role'$.
Let $\centroid^*$ be the ``intermediate'' centroids, where we have moved $v$ from $i$ to $j$ but we have not
updated the profiles of neighbors of $v$.  Then
\[
\begin{split}
    \gain{v, j} = 
    & - n_i\norm{\centroid_i}^2 + n_i'\norm{\centroid_i^*}^2 - n_j\norm{\centroid_j}^2 + n_j'\norm{\centroid_j^*}^2 \\
    & + \sum_{(v, w) \in E} 2\prof{w, \role}_i - 2\prof{w, \role}_j  - 2  \\
    & - \sum_{\ell = 1}^k 2d_\ell \spr{(\centroid_\ell^*)_j  - (\centroid_\ell^*)_i - d_\ell / n_\ell' }.  \\
\end{split}
\]
\end{proposition}

\begin{proof}[Proof of Proposition~\ref{prop:fastgain}]
Since $\centroid$ is the average of $x_i$,
a known identity $\sum (x_i - \centroid)^2 = \sum (x_i^2 - \centroid^2)$ allows us to 
rewrite the gain as
\[
\begin{split}
	\gain{v, j} = & \sum_{u \in V} \norm{\prof{u, \role}}^2 -  \norm{\prof{u, \role'}}^2 \\
	& - \sum_{\ell = 1}^k n_\ell \norm{\centroid_\ell}^2 -  n_\ell' \norm{\centroid_\ell'}^2.
\end{split}
\]
Let us define $B(w)$ to be a term in the first sum.
Let us abbreviate $\prof{w, \role}$ as $\prof{w}$.
Then $B(w) \neq 0$ if and only if $(v, w) \in E$,
and in such a case it is equal to
\[
\begin{split}
	B(w) & = \prof{w}_i^2 - (\prof{w}_i - 1)^2 + \prof{w}_j^2 - (\prof{w}_j + 1)^2 \\
	& = 2\prof{w}_i - 2\prof{w}_j  - 2.
\end{split}
\]
We now consider the second sum, which we will rewrite as
\[
	\sum_{\ell = 1}^k n_\ell \norm{\centroid_\ell}^2 -  n_\ell' \norm{\centroid_\ell^*}^2 +
	\sum_{\ell = 1}^k n_\ell' \norm{\centroid_\ell^*}^2 -  n_\ell' \norm{\centroid_\ell'}^2.
\]
The first sum, which we will denote by $A$ is equal to
\[
	A = n_i\norm{\centroid_i}^2 - n_i'\norm{\centroid_i^*}^2 + n_j\norm{\centroid_j}^2 - n_j'\norm{\centroid_j^*}^2. 
\]

Let us write $R(\ell)$ to be a term in the second sum. We can express this term as
\[
\begin{split}
	R(\ell) & = n_\ell'\spr{(\centroid_\ell^*)_i^2 - ((\centroid_\ell^*)_i - \gamma_\ell)^2 + (\centroid_\ell^*)_j^2 - ((\centroid_\ell^*)_j +  \gamma_\ell)^2} \\
	        & = 2d_\ell \spr{(\centroid_\ell^*)_i - (\centroid_\ell^*)_j -  \gamma_\ell}, \quad \text{where} \quad \gamma_\ell = d_\ell / n_\ell'.
\end{split}
\]
To conclude, we have
\[
	\gain{v, j} = - A + \sum_{(v, w) \in E} B(w) - \sum_{\ell = 1}^k R(\ell),
\]
which proves the identity.
\end{proof}



\begin{proof}[Proof of Proposition~\ref{prop:greedytime}]
According to Proposition~\ref{prop:fastgain}, computing the gain can be done in time $\bigO{k + \dg{v}}$. 
Consequently, computing $j^*$ requires time $\bigO{k^2 + k\dg{v}}$. 
Updating the profiles, the centroids, and the role assignment $\role$
can be done in time $\bigO{k + \dg{v}}$. 
Summing over all the vertices gives us a total running time of $\bigO{k^2 n + km}$.
\end{proof}

%% END

\begin{proof}[Proof of Proposition~\ref{prop:exacttime}]
At each iterations we increase the number of roles. Since the number of roles
is at most $n$, we can have at most $\bigO{n}$ iterations.
Next we analyze the running time of each iteration.

Let us represent the profiles with sorted sparse lists. The size of a profile
of a vertex $v$ is $\bigO{\dg{v}}$. The total time of constructing the profiles is
then
\[
	\bigO{\sum_{v \in V} \dg{v} \log \dg{v}} \subseteq \bigO{m \log n}.
\]

Grouping based on profiles can be done by sorting the profiles in
lexicographical order, and then comparing the consecutive profiles.

We next analyze the time needed to sort the profiles in lexicographical order.
Notice that comparing the profiles  $\prof{v}$ and $\prof{w}$ of two vertices $v$ and $w$
cannot be done in constant time;
instead time $\bigO{\min\{\dg{v}, \dg{w}\}}$ is required.

Assume that merge sort is used for ordering the profiles lexicographically.
Consider a counter $\Delta(v)$ defined as follows:
initially, $\Delta(v)$ is set to $0$.
Consider, that we are merging two sorted vertex lists during the merge sort.
Assume that we are comparing vertex $v$ from the first list and vertex $w$ from the second list,
and assume that we determine $\prof{v; \role} \leq \prof{w; \role}$.
In such case, we increase $\Delta(v)$ by $\dg{v}$.
If $\prof{v; \role} > \prof{w; \role}$, then we increase $\Delta(w)$ by $\dg{w}$.

The increase of $\Delta(v)$ or $\Delta(w)$ is an upper bound for the
time that is required for each comparison, so
it immediately follows that sorting can be done in time $\bigO{\sum_u \Delta(u)}$.

Note whenever we increase the counter, we always advance to the next vertex in
the list. That is,
each counter $\Delta(v)$ can be increased only once during a single merge,
and each vertex participates only in $\bigO{\log n}$ merges.
Consequently, $\Delta(v) \in \bigO{\dg{v} \log n}$.
This implies that sorting the profiles using  merge sort requires time
$\bigO{\sum_v \dg{v} \log n} = \bigO{m \log n}$.
\end{proof}



We will first prove Proposition~\ref{proposition:np}.

\begin{proof}[Proof of Proposition~\ref{proposition:np}]
We will prove the hardness from \tmatch: an \np-hard problem where we are given
a universe $U$ and family $\mathcal{M}$ of sets of size $3$, and we are asked
to find a maximum disjoint cover. 

Assume that we are given an instance $(U, \mathcal{M})$ of \tmatch.  Let $n =
\abs{U}$ and $m = \abs{\mathcal{M}}$. We can safely assume that $n$ is divisible by 3.

Define $h = {n \choose 3}$ the number of possible
sets of size $3$ over $U$. Let $\ell = {n - 1 \choose 2}$ be the number of possible
sets of size $3$ containing a fixed vertex in $U$.
Let us define $t = 8m$ and $s = 2t$.

\emph{Graph construction:}
The graph constists of two major parts.

The first part entails 7 vertex sets: we start with $A_1$ and $A_2$ with $\abs{A_1} = 3$ and $\abs{A_2}
= \ell - 1$.
A single vertex, say $v$, 
is connected to an additional vertex, say $w$. We write $A_3 = \set{v}$ and $A_4 = \set{w}$.
Futhermore, $w$ is connected to a biclique of $A_5 = K(s, s)$.
%Each vertex $a \in A_1$ is connected to its own clique of size $h$; the vertices in these cliques form $A_6$.
%
Each vertex $u \in A_1$ and $v \in A_2$ is connected with a \emph{fat} edge:
$t$ copies of a path $u$--$x$--$v$, where $x$ is a vertex unique to the path.
These vertices form $A_6$.
Similarly, each vertex $u \in A_1$ and $v \in A_3$ is connected with a \emph{fat} edge;
the intermediate vertices form $A_7$.
%
%Finally, one
%vertex, say $b^* \in B$ is connected to an additional vertex, who is also connected
%to a biclique $K(s, s)$.
%
The final graph consists of $t$ copies of the first part. We redefine $A_i$ to be
the union of the corresponding groups in each copy.

The second part is very similar to the first.  It entails the vertex sets
$B_1$, $B_2$, and $B_3$ with $\abs{B_1} = n$, $\abs{B_2} = h - m$ $\abs{B_3} = m$.
Here $B_1$ corresponds to the universe $U$, and $B_2$ and $B_3$ correspond to
triplets of elements in $U$. Each vertex in $B_3$ is connected to its own vertex.
These vertices form $B_4$. Each vertex in $B_4$ is also connected to its own dedicated biclique
$K(s, s)$. The vertices in bicliques form $B_5$.
A vertex in $B_2$ is connected to the corresponding vertices in $B_1$ with a fat edge
of size $t$. The intermediate vertices form $B_6$.
Similarly, a vertex in $B_3$ is connected to the corresponding vertices in $B_1$ with a fat edge
of size $t$. The intermediate vertices form~$B_7$.



\iffalse
The first part entails vertex sets $A$ and $B$ with $\abs{A} = 3$ and $\abs{B}
= \ell$.  Each vertex $a \in A$ and $b \in B$ is connected with a \emph{fat} edge:
$t$ copies of a path $a$--$x$--$b$, where $x$ is a vertex unique to the path.
Each vertex $a \in A$ is connected to its own clique of size $h$.  Finally, one
vertex, say $b^* \in B$ is connected to an additional vertex, who is also connected
to a biclique $K(s, s)$.

The final graph consists of $t$ copies of the first part.

The second part is very similar to the first.  It entails the vertex sets $A'$
and $B'$ with $\abs{A'} = n$ and $\abs{B'} = h$. Here $A'$ corresponds to the
universe $U$.  Each vertex $a \in A'$ and $b \in B'$ is connected with a
\emph{fat} edge of size $t$.  Finally, let $C'$ be the subset of $B'$ that
corresponds to the sets in $\mathcal{M}$.  Each vertex $c \in C'$ is connected
to its own unique vertex. We will denote these vertices by $O'$.
Each vertex in $O'$ is also connected to its own dedicated bi-clique
$K(s, s)$. 
\fi

\emph{Role assignment:}
Let us now define a role assignment, later this will turn out to be the optimal
assignment. Let $\mathcal{W}$, not necessarily a subset of $\mathcal{M}$, be a
matching covering completely~$U$.

Let $C_3$ be the vertices in $B_2 \cup B_3$ corresponding to $\mathcal{W}$.
Let $C_1 = B_1$, $C_2 = (B_2 \cup B_3) \setminus C_3$ $C_4 = B_4$, $C_5 = B_5$.
Let $C_6$ be the vertices in $B_6 \cup B_7$ adjacent to $C_3$, and set $C_7 = (B_6 \cup B_7) \setminus C_6$.
We define $\role(v) = i$, where $v \in A_i \cup C_i$.
Let us write $\role_{\mathcal{W}} = \role$.

Let us compute the cost of $\role$.
To that end, write $c_{ij}$ to be the cost incurred by the $j$th component of the $i$th role.
The only non-zero costs are $c_{24}$, $c_{42}$, $c_{34}$, and $c_{43}$.
Define $n_i = \abs{\set{v ; \role(v) = i}}$,
and let $\alpha = \abs{A_3 \cup (B_3 \cap C_3)}$ and $\beta = \abs{B_3 \cap C_2}$.
We can express the cost as
\[
	c_{24} + c_{34} + c_{42} + c_{43} = f(\beta, n_2) + f(\alpha, n_3) + 2f(\alpha, n_4),
	%c_{24} = f(\beta, n_2)
	%c_{34} = f(\alpha, n_3)
	%c_{42} = f(\alpha, n_4)
\]
where $f(x, y) = x(y - x)/y$.
The counts $n_i$ do not depend on $\mathcal{W}$.
Since $\alpha + \beta = n_4$, so the cost of depends on $\mathcal{W}$ only via $\alpha$.
We have $2\alpha \geq 2t \geq n_3, n_4$ and $2\beta \leq 2m \leq t \leq n_2$.
Since $f(x, y)$ is a parabola peaking at $x = y / 2$,
the cost decreases as $\alpha$ increases.

Since $f(x, y) \leq \min(x, y / 2 - x)$, we can upper-bound the cost
by $n_3 + 2n_4 - 3\alpha + \beta < 4m$.

\iffalse

Let $\alpha = \abs{A_3 \cup C_3}$ be the number of vertices with $\role(v) = 3$.
Let $\beta = \abs{A_2 \cup C_2}$ be the number of vertices with $\role(v) = 2$.
Define also $\alpha_1 = \abs{A_3 \cup (B_3 \cap C_3)}$,
and $\beta_1 = \abs{B_3 \cap C_2}$.
A laborous comptuation reveals that the cost of $\role$ is equal to
\begin{equation}
\label{eq:cost}
	f(\alpha_1, \alpha) + f(\beta_1, \beta) + 2f(\alpha_1, \alpha_1 + \beta_1),
\end{equation}
where $f(x, y) = 2x(y - x)/y$.
Note that $\alpha$, $\beta$, and $\alpha_1 + \beta_1 = \abs{A_2} + \abs{B_3}$ are all constant.
Moreover, since
$\alpha_1 \geq \alpha / 2$,
$\beta_1 \leq \beta / 2$,
$\alpha_1 \geq \beta_1$, Eq.~\ref{eq:cost} decreases as $\abs{B_3 \cap C_3}$ increases.

Since $f(x, y) \geq x$ for $x \leq y / 2$,
we can upper-bound the cost by
\[
\begin{split}
	\alpha - \alpha_1 + \beta_1 + 2\beta_1 & = \abs{A_3 \cup C_3} - \abs{A_3 \cup (B_3 \cap C_3)} + 
	t + n / 3 - (t + \abs{D}) + 3(m - \abs{D}) \\
	& = n / 3 + 3m - 4\abs{D} < 4m.
\end{split}
\]

\emph{Role assignment:}
Let us now define a role assignment, this will turn out to be optimal
assignment.  Let $\mathcal{W} \subseteq \mathcal{U}$ be the maximum matching.
If $\mathcal{W}$ does not cover completely $\mathcal{U}$, then let $\mathcal{W}'$
be additional sets outside $\mathcal{M}$ completing the matching.
Let $D$ be the vertices in $B'$ corresponding to $\mathcal{W}$.
Let $D'$ be the vertices in $B'$ corresponding to $\mathcal{W}'$.
Let $K$ be the union of all bi-cliques.
\[
	\role(v) =
\begin{cases}
1 & v \in K,\\
2 & v \text{ adjacent to } K,\\
3 & v \in A \text{ or } v \in A',\\
4 & v \in D \cup D', \text{ or } v = b^*, \\
5 & v \in B' \setminus (D \cup D'), \text{ or } v \in B \setminus \set{b^*},\\
6 & v \text{ is a fat edge adjacent to } w, \role(w) = 4,\\
7 & v \text{ is a fat edge adjacent to } w, \role(w) = 5.\\
\end{cases}
\]

Define the roles as follows: The roles of vertices in bi-cliques $K(s, s)$ are
set to 2, the roles of adjacent vertices to the bi-cliques are set to 1. The
roles of vertices in $A$ and $A'$ are set to 3. The roles of vertices
corresponding to $\mathcal{W}$, as well as $b^*$, are set to $4$.
The roles for the remaining vertices in $B$ and $B'$ are set to 

Let us compute the cost of $\role$.
Let $\alpha = t + n / 3$ be the number of vertices with $\role(v) = 4$.
Among these vertices,
let $\alpha_1 = t + \abs{D}$ be the number of vertices adjacent to a vertex with
a role of 2.
Let $\beta = t\ell + k - \alpha$ be the number of vertices with $\role(v) = 5$.
Among these vertices,
let $\beta_1 = m - \abs{D}$ be the number of vertices adjacent to a vertex with
a role of 2.
A laborous comptuation reveals that the cost of $\role$ is equal to
\begin{equation}
\label{eq:cost}
	f(\alpha_1, \alpha) + f(\beta_1, \beta) + 2f(\alpha_1, \alpha_1 + \beta_1),
\end{equation}
where $f(x, y) = 2x(y - x)/y$. Note that $\alpha$, $\beta$, and $\alpha_1 + \beta_1$ are all constant.
Moreover, since
$\alpha_1 \geq \alpha / 2$,
$\beta_1 \leq \alpha / 2$,
$\alpha_1 \geq \beta$, Eq.~\ref{eq:cost} decreases as $\abs{D}$ increases.


Since $f(x, y) \geq x$ for $x \leq y / 2$,
we can upper-bound the cost by
\[
\begin{split}
	\alpha - \alpha_1 + \beta_1 + 2\beta_1 & = t + n / 3 - (t + \abs{D}) + 3(m - \abs{D}) \\
	& = n / 3 + 3m - 4\abs{D} < 4m.
\end{split}
\]

\fi

\emph{Role $\role$ is optimal:}
Let $\role^*$ be the optimal role assignment with a cost of $\sigma$.  Consider
that if instead of selecting optimal centroids for $\role^*$, we select the
optimal centroids---we denote them by $\mu'$---among the profiles of the
vertices in the cluster. We know that the cost of $\role^*$ w.r.t. $\mu'$, say
$\tau$, is at most $2\sigma$. If $\tau \geq 8m$, then $\sigma \geq 4m$, the cost of $\role$, which
violates the optimality of $\role^*$. Consequently, $\tau < 8m = t$.

Note that $\mu'$ are all integral. We can safely assume that the copies of the
first part of the graph have the same role assignment. This immediately implies
that the profiles of $\role^*$ should match \emph{exactly} the centroids $\mu'$.

There are 6 groups with distinct degrees in the first part:
$\dg{u} = t\ell$ for $u \in A_1$,
$\dg{u} = 3t$ for $u \in A_2$,
$\dg{u} = 1 + 3t$ for $u \in A_3$,
$\dg{u} = 2s + 1$ for $u \in A_4$,
$\dg{u} = s + 1$ for $u \in A_5$,
$\dg{u} = 2$ for $u \in A_6 \cup A_7$.
Since the vertices from different groups have different degrees, $\role^*$
must be a refinement of this partition.
Moreover, $A_6$ connects to $A_2$ and $A_7$ connects to $A_3$.
Thus they cannot have the same role. 
This gives us 7 groups, and $\role^*$ must be a refinement of this partition.
Since we have only 7 roles, this must be the role assignment.

Let us now consider the second part of the graph.
Let $u \in B_1$.
The only integral centroid that matches $\dg{u}$ is $\mu'_1$, and the
remaining centroids differ in degree
by at least of $t$, consequently $\dist{u, \mu'_i} \geq t$ for $i \neq 3$.
This forces, $\role^*(u) = 1$. Similarly,
$\role^*(u) = 4$, for $u \in B_4$, 
$\role^*(u) = 5$, for $u \in B_5$,
$\role^*(u) = 6, 7$, for $u \in B_6 \cup B_7$.

Let $u \in B_2 \cup B_3$. If $\role^*(u) \neq 2, 3$, then each vertex in a fat edge will
introduce a cost of at least 1, when compared to the integral centroids $\mu'$.
Since there are $t$ of these vertices, we must have $\role^*(u) = 2, 3$.  Using
a similar argument, among the vertices in $B_2 \cup B_3$ adjacent (by a fat edge) to a vertex $v \in B_1$,
only one has the role $3$, the remaining adjacent vertices have the role $2$.

This shows that $\role^* = \role_{\mathcal{W}}$, where $\mathcal{W}$ are the sets corresponding
to the vertices with role $3$. We saw earlier that the cost is minimized when $\alpha = t +
\abs{\mathcal{W} \cap \mathcal{M}}$
is maximized.
\end{proof}


\iffalse
\emph{(i)} a vertex $o$ with $\dg{o} = 2s + 1$,
\emph{(ii)} a vertex in $K(s, s)$ with $\dg{o} = s + 1$,
\emph{(iii)} vertices $a \in A$ with $\dg{a} = t\ell$,
\emph{(iv)} vertex $b^*$ with $\dg{b^*} = 1 + 3t$,
\emph{(v)} the remaining vertices $b \in B \setminus \set{b^*}$ with $\dg{b} = 3t$, and finally
\emph{(vi)} the vertices in fat edges with degree of $2$.
Since the vertices from different groups have different degrees, $\role^*$
must be a refinement of this partition. Note that the fat edges can be split further into
two groups:
\emph{(vi.a)} vertices connecting to $b^*$, and
\emph{(vi.b)} vertices connecting to $B \setminus \set{b^*}$.
This gives us 7 groups, and $\role^*$ must be a refinement of this partition.
Since we have only 7 roles, this must be the role assignment.


Let us now consider the second part of the graph.
Let $a \in A'$.
The only integral centroid that matches $\dg{a}$ is $\mu'_3$, and the
remaining centroids differ in degree
by at least of $t$, consequently $\dist{a, \mu'_i} \geq t$ for $i \neq 3$.
This forces, $\role^*(a) = 3$. Similarly, each vertex $o \in O'$ has
$\role^*(o) = 1$, and the vertices in the bi-cliques have the role 2.
Each vertex $x$ in a fat edge must have $\role^*(x) = 6$ or $\role^*(x) = 7$.

Let $b \in B'$. If $\role^*(b) \neq 4, 5$, then each vertex in a fat edge will
introduce a cost of at least 1, when compared to the integral centroids $\mu'$.
Since there are $t$ of these vertices, we must have $\role^*(b) = 3, 4$.  Using
a similar argument, among the vertices in $B'$ adjacent to a vertex $a \in A'$,
only one has the role $4$, the remaining adjacent vertices have the role $5$.


The cost of $\role^*$ is now lower-bounded by the cost given in
Eq.~\ref{eq:cost}, and it is equal if and only if $\role^* = \role$.

Since the cost of $\role$ decreases as $\abs{D}$ increases, the optimal $\role$
will correspond to the optimal matching.
\end{proof}
\fi

To prove Proposition~\ref{proposition:np-fixed},
we need to introduce the following \np-complete problem.

\begin{problem}[\tuples]
Assume a universe $U$ and a set $\mathcal{S}$ of 5-tuples over $U$,
such that each $u \in U$ occurs in at least 3 tuples.
Is there a subset $B \subseteq U$ such that each exactly 2 entries in each $S \in \mathcal{S}$
belong to $B$.
\end{problem}

\begin{proposition}
\tuples is \np-complete.
\end{proposition}

\begin{proof}
\tuples is obviously in \np. We prove the hardness by reduction from \tsat.

Assume that we are given $m$ clauses $C_1, \ldots, C_m$ using $n$ variables, in
total. We can safely assume that each clause contains exactly 3 variables by
allowing the repetition of a variable.

We define the universe $U$ to contain $2n + 2 + 2m$ variables, which we will
denote by $t$, $f$, $x_1, \ldots, x_n$, $y_1, \ldots, y_n$,
$d_1, \ldots, d_m$, and $e_1, \ldots, e_m$.
To simplify notation,
we will identify $x_i$ with the $i$th positive literal and $y_i$ with the $i$th negative literal. 

We introduce the following clauses:
\emph{(1)} $(t, t, f, f, f)$,
\emph{(2)} $(x_i, \neg x_i, t, f, f)$, where $i = 1, \ldots, n$,
\emph{(3)}
$(\neg c_1, \neg c_2, \neg c_3, d_j, e_j)$, where $C_j = c_{1} \lor c_{2} \lor c_{3}$
is the $j$th clause.

To make sure that each variable occurs in at least 3 tuples, we copy each tuple
3 times.


Assume there is $B$ solving \tuples. Set the $i$th variable to be true if $x_i \in B$, and false otherwise.
At most $2$ variables $\neg c_i$ are in $B$, for $C_j = c_{1} \lor c_{2} \lor c_{3}$.
Assume that $\neq c_1 = \notin B$.
Since either $x_i \in B$ or $\neg x_i \in B$, this forces $c_1 \in B$, 
making $C_j$ satisfied.

Now assume that we can satisfy each clause. First set $B = \set{t}$. Add
$x_i$ into $B$ if the $i$th variable is true, otherwise insert $\neg x_i$.
Since a clause $C_j = c_{1} \lor c_{2} \lor c_{3}$ is satisfied, we have at most 2 entries in $B$ among $\neg c_i$. 
Insert $d_j$ and/or $e_j$ to $B$ so that $(\neg c_1, \neg c_2, \neg c_3, d_j, e_j)$ contains exactly 2 entries in $B$.
The resulting $B$ solves \tuples.
\end{proof}

\begin{proof}[Proof of Proposition~\ref{proposition:np-fixed}]
The problem is in \np.  We will prove the hardness from \tuples.
Assume that we are given an instance $(U, \mathcal{S})$ of \tuples.
Let $n = \abs{U}$ and $m = \abs{\mathcal{S}}$.
Let $\ell_u$ be the total number of occurrences, counting multiplicities,
of $u$ in $\mathcal{S}$.

Define the graph as follows:
For each $u \in U$, add a cycle of length $\ell_u$. Note that the cycle is simple
since $\ell_u \geq 3$. Denote this cycle by $D_u$.

For each tuple $C_j$ add a vertex $w_j$, and connect it to 
a vertex in $D_u$, where $u \in C_j$. The connection can and should be done such
that each vertex in each cycle is connected to exactly one $w_j$.

Set the number of roles $k = 3$, and define the following centroids $\mu_1 = (2, 0, 1)$, $\mu_2 = (0, 2, 1)$,
and $\mu_3 = (2, 3, 0)$.

Let $r$ be the optimal role assignment for the defined centroids.
The cost of is 0 if and only if
\emph{(i)} every vertex in $v \in D_u$ has the same role, $\role(v) = 1, 2$,
\emph{(ii)} $\role(w_j) = 3$,
\emph{(iii)} for each $w_j$ there are two adjacent vertices with role 1
and three adjacent vertices with role 2.

Consequently, $B$ in \tuples corresponds to the vertices with role 1.
\end{proof}
