\section{Related work}
\label{section:related}


Role-discovery methods can be broadly classified into three categories~\cite{rossi2015role}: 
($i$) graph-based, 
($ii$) feature-based, and 
($iii$) hybrid.
Graph-based methods compute roles directly from the graph representation. 
A number of different definitions have been suggested
for quantifying when two nodes are equivalent and should be assigned to the same role. 

In automorphic equivalence (see, for example,~\cite{hanneman:05:introduction}),
where two vertices $u$ and $v$ are equivalent if there is graph automorphism
mapping $u$ to $v$.  Furthermore, the vertices are structurally
equivalent~\cite{lorrain1971structural}, if the automorphism does not alter the
remaining vertices. A more relaxed definition of equivalence is regular
equivalence~\cite{everett1994regular}, where vertices are equivalent if they
have equivalent neighbors, ignoring any multiplicities. 

Blockmodelling can be viewed as a role assignment task. Here the idea is to
model the edge appearance between two vertices $u$ and $v$ based on their roles.
The roles are viewed as latent variables, and are learned using standard
statistical optimization techniques~\cite{snijders:97:estimation}.
For more details on blockmodels, see the survey of Goldenberg et al.~\cite{Goldenberg:2010:SSN}.

\iffalse
stochastic equivalence~\cite{holland1981exponential}.
Typically, methods for graph-based role discovery rely on 
blockmodels~\cite{airoldi2009mixed,holland1983stochastic}, 
which are based on matrix-decomposition techniques
and are not easily scalable to very large graphs. 
\fi

Feature-based role discovery is a more modern approach that relies on
representing each node in the graph by a feature vector and assigning to the
same role nodes with similar feature
vectors~\cite{henderson2012rolx,rossi2015role,rossi2013modeling,zhao2013inferring,gilpin2013guided}.
Features can be extracted from graph-based properties of each node, such as,
degree, clustering coefficient, centrality, etc., or combined with other
information that may be available for the graph nodes, such as dynamic behavior
or node attributes. 
Once feature vectors are constructed, the assignment
problem can be viewed as a traditional clustering problem, and thus can be
approached with classic clustering techniques. From technical point of view,
our setting differs fundamentally since our features, namely the roles of
neighbors, depend on the clustering.
Hybrid role discovery methods combine both methods. For example, using learned blockmodels
as features~\cite{rossi2015role}.

The typical key component in role discovery is how to measure similarity
between two vertices. The simplest way to do this is by computing a distance
between the neighbors, such as, cosine similarity or Pearson coefficient of
common neighbors.  As a more intricate example, we can also based vertex
similarity on features of neighboring vertices~\cite{yang2015network} or
spectral analysis~\cite{tsourakakis2014toward}.


We should stress that role discovery has a different goal than community
detection: in community detection, we are interested in finding highly
connected subsets, whereas in role discovery we want to find vertices that
serve similar purpose. Interestingly, Ruan and Parthasarathy~\cite{ruan2014simultaneous}
proposed mining roles and communities simultaneously.

Finally, a fruitful direction for role discovery is to adapt existing
methodology designed for static networks for dynamic settings, where the role
may change over time~\cite{abnar2015ssrm,li2013learning}. We leave adapting
our approach for dynamic settings as future work.
