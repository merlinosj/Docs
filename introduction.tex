\section{Introduction}


Modeling interconnected entities as graphs (or networks)
allows us study the global structure and function of a system,
instead of looking at single entities in isolation.
The understanding of the structure of a network in a holistic way
can be further supported by our ability to 
understand the \emph{role} of a single vertex 
with respect to its local neighborhood, 
or with respect to the whole network.
Accordingly, 
\emph{role discovery} 
has emerged as an important
graph-mining task~\cite{gilpin2013guided,danilevsky2013entity,henderson2012rolx,rossi2015role,ruan2014simultaneous,yang2015network,zhao2013inferring}, 
together with other standard graph-mining problems, 
such as community detection, link prediction,  etc.

Role discovery can be a valuable tool for exploratory graph mining. 
For instance, identifying the role of person in a social network 
may provide cues for understanding the social behavior of the person
in relation to her peers. 
Similarly, identifying the role of a vertex in a technological network 
may give useful information about the function of the vertex in the network, 
or it may be used to detect anomalies~\cite{rossi2013multi}.
In fact, Rossi and Ahmed~\cite{rossi2015role}
provide an extensive and well-documented list of graph-mining tasks
that can be facilitated by role discovery. 
The list includes applications such as 
classification, active learning, graph visualization, 
transfer learning, graph compression, entity resolution, and more.

% Not really needed, if we are low on space
Role discovery can be seen as a process that
partitions the vertices of a graph into equivalence classes.
Nodes in the same class are assigned to the same role. 
Equivalence classes are typically
aiming at capturing the structural relation between the vertices and their neighbors. 
In this way roles can represent 
structural patterns in the graph, 
such as, star centers, bridges, peripheral vertices, vertices in near-cliques, etc.~\cite{henderson2012rolx}.

Many different approaches have been proposed
for defining when two vertices should be considered equivalent and, thus,  
should be assigned to the same role. 
Some of the first methods proposed in mathematical sociology
rely on identifying structural or automorphic equivalence 
classes~\cite{everett1994regular,lorrain1971structural}, 
while newer methods 
represent vertices by feature vectors 
and assign to the same role vertices with similar 
feature vectors~\cite{henderson2012rolx,rossi2015role,rossi2013modeling,zhao2013inferring}.

An attractive definition for 
discovering roles in networks is based on the concept of
\emph{regular equivalence}~\cite{everett1994regular,white1983graph}:
According to regular equivalence, 
two vertices should be assigned to the same role
only if their neighbors have the same roles, ignoring multiplicities.
For example, 
for the collaboration network of a company
we may discover 
that vertices with role $A$ (``project manager'')
are connected to vertices with role 
$B$ (``business analyst'') and $C$ (``s/w developer''), 
while vertices with roles $B$ and $C$ are typically not connected to each other. 

In this paper we present a new approach to role discovery, 
inspired by the definition of regular equivalence.
As the original definition is too strict to be of use in real-world datasets, 
we provide a \emph{relaxation} that provides robustness and can tolerate noise in the data.
In particular, we define an objective function that quantifies
the degree to which a given role assignment 
satisfies regular equivalence.
Given a target number of roles $k$
we then ask to find the role assignment that minimizes our objective function.
We also take multiplicities into account: a vertex with 100 neighbors having role $A$
is treated differently than a vertex with a single neighbor having role $A$.

\begin{figure}
\begin{tikzpicture}[baseline = (current bounding box.south)]
\node[graphcl3, graphnode] (n0) at (-0.195230in, -0.629600in) {$\scriptscriptstyle 0$};
\node[graphcl2, graphnode] (n1) at (0.237355in, -0.458975in) {$\scriptscriptstyle 1$};
\node[graphcl2, graphnode] (n4) at (-0.202680in, -0.378515in) {$\scriptscriptstyle 4$};
\node[graphcl1, graphnode] (n3) at (0.055400in, -0.662550in) {$\scriptscriptstyle 3$};
\node[graphcl1, graphnode] (n5) at (-0.495860in, -0.631050in) {$\scriptscriptstyle 5$};
\node[graphcl1, graphnode] (n9) at (0.056340in, -0.331475in) {$\scriptscriptstyle 9$};
\node[graphcl1, graphnode] (n12) at (-0.627850in, -0.402875in) {$\scriptscriptstyle 12$};
\node[graphcl1, graphnode] (n17) at (-0.132200in, -0.855000in) {$\scriptscriptstyle 17$};
\node[graphcl1, graphnode] (n26) at (-0.429895in, -0.858850in) {$\scriptscriptstyle 26$};
\node[graphcl2, graphnode] (n2) at (0.192100in, -0.057610in) {$\scriptscriptstyle 2$};
\node[graphcl2, graphnode] (n13) at (0.446975in, -0.093245in) {$\scriptscriptstyle 13$};
\node[graphcl1, graphnode] (n20) at (0.481320in, -0.849000in) {$\scriptscriptstyle 20$};
\node[graphcl2, graphnode] (n6) at (-0.304685in, 0.139655in) {$\scriptscriptstyle 6$};
\node[graphcl3, graphnode] (n7) at (0.286355in, 0.318150in) {$\scriptscriptstyle 7$};
\node[graphcl2, graphnode] (n8) at (-0.172725in, 0.302300in) {$\scriptscriptstyle 8$};
\node[graphcl2, graphnode] (n18) at (-0.087970in, -0.095142in) {$\scriptscriptstyle 18$};
\node[graphcl1, graphnode] (n25) at (0.554350in, -0.342300in) {$\scriptscriptstyle 25$};
\node[graphcl1, graphnode] (n31) at (0.676500in, -0.012445in) {$\scriptscriptstyle 31$};
\node[graphcl2, graphnode] (n11) at (-0.570100in, 0.070990in) {$\scriptscriptstyle 11$};
\node[graphcl1, graphnode] (n23) at (-0.379895in, 0.573350in) {$\scriptscriptstyle 23$};
\node[graphcl1, graphnode] (n28) at (-0.455200in, -0.091015in) {$\scriptscriptstyle 28$};
\node[graphcl1, graphnode] (n32) at (-0.455200in, -0.291015in) {$\scriptscriptstyle 32$};
\node[graphcl1, graphnode] (n29) at (-0.528800in, 0.497080in) {$\scriptscriptstyle 29$};
\node[graphcl1, graphnode] (n19) at (0.578350in, 0.362865in) {$\scriptscriptstyle 19$};
\node[graphcl1, graphnode] (n14) at (0.681150in, 0.180865in) {$\scriptscriptstyle 14$};
\node[graphcl1, graphnode] (n22) at (0.257470in, 0.803250in) {$\scriptscriptstyle 22$};
\node[graphcl1, graphnode] (n15) at (0.574150in, 0.648200in) {$\scriptscriptstyle 15$};
\node[graphcl1, graphnode] (n24) at (0.428340in, 0.746050in) {$\scriptscriptstyle 24$};
\node[graphcl1, graphnode] (n16) at (0.050615in, 0.571950in) {$\scriptscriptstyle 16$};
\node[graphcl1, graphnode] (n10) at (0.697500in, 0.513900in) {$\scriptscriptstyle 10$};
\node[graphcl1, graphnode] (n30) at (0.040093in, 0.397720in) {$\scriptscriptstyle 30$};
\node[graphcl1, graphnode] (n33) at (-0.247890in, 0.690400in) {$\scriptscriptstyle 33$};
\node[graphcl1, graphnode] (n34) at (-0.635250in, 0.336005in) {$\scriptscriptstyle 34$};
\node[graphcl1, graphnode] (n35) at (-0.995000in, -0.019115in) {$\scriptscriptstyle 35$};
\node[graphcl1, graphnode] (n27) at (-0.970750in, 0.285785in) {$\scriptscriptstyle 27$};
\node[graphcl1, graphnode] (n21) at (0.917700in, -0.269635in) {$\scriptscriptstyle 21$};
\begin{pgfonlayer}{background}
\draw[graphedge] (n0) edge (n1);
\draw[graphedge] (n0) edge (n4);
\draw[graphedge] (n0) edge (n3);
\draw[graphedge] (n0) edge (n5);
\draw[graphedge] (n0) edge (n9);
\draw[graphedge] (n0) edge (n12);
\draw[graphedge] (n0) edge (n17);
\draw[graphedge] (n0) edge (n26);
\draw[graphedge] (n1) edge (n4);
\draw[graphedge] (n2) edge (n1);
\draw[graphedge] (n1) edge (n13);
\draw[graphedge] (n1) edge (n20);
\draw[graphedge] (n4) edge (n5);
\draw[graphedge] (n4) edge (n6);
\draw[graphedge] (n4) edge (n8);
\draw[graphedge] (n9) edge (n13);
\draw[graphedge] (n9) edge (n18);
\draw[graphedge] (n2) edge (n3);
\draw[graphedge] (n2) edge (n13);
\draw[graphedge] (n2) edge (n6);
\draw[graphedge] (n2) edge (n7);
\draw[graphedge] (n2) edge (n8);
\draw[graphedge] (n2) edge (n18);
\draw[graphedge] (n2) edge (n25);
\draw[graphedge] (n2) edge (n31);
\draw[graphedge] (n13) edge (n25);
\draw[graphedge] (n13) edge (n19);
\draw[graphedge] (n13) edge (n14);
\draw[graphedge] (n13) edge (n21);
\draw[graphedge] (n7) edge (n6);
\draw[graphedge] (n6) edge (n18);
\draw[graphedge] (n11) edge (n6);
\draw[graphedge] (n6) edge (n23);
\draw[graphedge] (n6) edge (n28);
\draw[graphedge] (n6) edge (n29);
\draw[graphedge] (n7) edge (n13);
\draw[graphedge] (n7) edge (n8);
\draw[graphedge] (n7) edge (n19);
\draw[graphedge] (n7) edge (n14);
\draw[graphedge] (n7) edge (n22);
\draw[graphedge] (n7) edge (n15);
\draw[graphedge] (n7) edge (n24);
\draw[graphedge] (n7) edge (n16);
\draw[graphedge] (n7) edge (n10);
\draw[graphedge] (n7) edge (n30);
\draw[graphedge] (n8) edge (n18);
\draw[graphedge] (n11) edge (n8);
\draw[graphedge] (n8) edge (n16);
\draw[graphedge] (n8) edge (n33);
\draw[graphedge] (n8) edge (n34);
\draw[graphedge] (n18) edge (n28);
\draw[graphedge] (n18) edge (n32);
\draw[graphedge] (n30) edge (n18);
\draw[graphedge] (n11) edge (n12);
\draw[graphedge] (n11) edge (n18);
\draw[graphedge] (n11) edge (n34);
\draw[graphedge] (n11) edge (n35);
\draw[graphedge] (n11) edge (n27);
\end{pgfonlayer}

\end{tikzpicture}\hspace{-4mm}
\begin{tikzpicture}
\begin{axis}[xlabel={role}, ylabel= {degree},
	width = 2.1cm,
	height = 2.2cm,
	ymin = 1,
	cycle list name=yaf,
	every node near coord/.append style = {font = \scriptsize, text = black, anchor = west},
	xtick = {1, 2, 3},
	ytick = {1, 2, 3, 5, 7, 8, 9, 12},
	scatter/classes = {1={yafcolor1},2={yafcolor3},3={yafcolor2}},
	scale only axis
    ]
\addplot+[only marks, no markers, point meta = explicit, nodes near coords]
	table[x index = 0, y index = 1, header = false, meta index = 2] {grooming_degree.dat};
\addplot+[only marks, mark = *, scatter, scatter src = explicit symbolic]
	table[x index = 0, y index = 1, header = false, meta index = 0] {grooming_degree.dat};

\pgfplotsextra{\yafdrawaxis{1}{3}{1}{12}}
\end{axis}

%{0: [0.0, 0.88461538461538458, 0.53846153846153844], 1: [2.875, 3.5, 0.75], 2: [7.0, 3.0, 0.0]}
\node[font = \scriptsize, inner sep = 0pt, anchor = north west] (lab) at (-0.15, -0.85) {Centroids};
\node[graphcl1, graphnode, below = 12pt of lab.west, anchor = west] (l1) {};
\node[graphcl2, graphnode, below = 1pt of l1] (l2) {};
\node[graphcl3, graphnode, below = 1pt of l2] (l3) {};

\node[right = 0pt of l1, font = \scriptsize] (c10) {:};
\node[right = 0pt of l2, font = \scriptsize] (c20) {:};
\node[right = 0pt of l3, font = \scriptsize] (c30) {:};

\node[right = 20pt of l1, font = \scriptsize, anchor = east] (c11) {0};
\node[right = 40pt of l1, font = \scriptsize, anchor = east] (c12) {0.93};
\node[right = 60pt of l1, font = \scriptsize, anchor = east] (c13) {0.54};

\node[right = 20pt of l2, font = \scriptsize, anchor = east] (c21) {3};
\node[right = 40pt of l2, font = \scriptsize, anchor = east] (c22) {3.5};
\node[right = 60pt of l2, font = \scriptsize, anchor = east] (c23) {0.75};

\node[right = 20pt of l3, font = \scriptsize, anchor = east] (c31) {7};
\node[right = 40pt of l3, font = \scriptsize, anchor = east] (c32) {3};
\node[right = 60pt of l3, font = \scriptsize, anchor = east] (c33) {0};
\end{tikzpicture}


\caption{Groom network of Rhesus Macaques~\cite{beisner:11:instability}.
3 roles are assigned using \alggreedy initialized by \alginitdeg. The scatter plot shows the degree of a
vertex as a function of its role.}
\label{figure:grooming}
\end{figure}


The proposed objective function is based on creating a \emph{profile} for each vertex, 
which represents the number of neighbor vertices for each other role.
Thus, vertices with the same role should have similar profiles.
This requirement is expressed as a $k$-means-type squared-error function. 
The approach resembles feature-based methods,
however, the important difference is the recursive nature of our definition: 
\emph{roles depend on profiles and profiles depend on roles}. 

An example of the roles discovered in a grooming network
of monkeys, Rhesus Macaques, 
is shown in Figure~\ref{figure:grooming}. 
In this example we search for $k=3$ roles. 
The role assignment is depicted with different colors, 
and the profile centroids for each role are shown in the bottom-right subplot.
We see that the first role ({\tt purple}) corresponds to relatively isolated individuals, 
while the other two roles ({\tt green} and {\tt orange}) correspond to more central ones. 
Observe that the {\tt green} role is indeed different than the {\tt orange} role,
as the individuals of the {\tt orange} role are connected to more individuals of the {\tt purple} role, 
and they are not connected to each other. 
In the upper-right subplot we show a scatter-plot of role vs.\ degree. 
We see that one of the two vertices with {\tt orange} role
has smaller no larger degree than five of the vertices with {\tt green} role, 
indicating that the role assignment we discover cannot be explained solely by degree. 


Our technical contributions are as follows: we formulate the optimization
problem and demonstrate that this problem is \np-hard. Furthermore, we show
that if we fix the profile centroids, 
the problem still remains \np-hard, and 
cannot be approximated.
On the positive side, we show that discovering a \emph{perfect}
role assignment, that is, a role assignment with $0$ cost, with smallest number
of roles $k$ can be done efficiently in polynomial time. We further propose two
natural heuristic algorithms for minimizing the cost function when $k$ is fixed: 
($i$) the first method is
a greedy hill-climbing algorithm, where we optimize a role for a single
vertex while keeping the remaining vertex roles constant,
($ii$) in the second approach we first fix the profiles, transforming
the problem into a standard clustering problem, that we solve using $k$-means algorithm,
and compute the new profiles from the obtained clustering.

We have benchmarked the proposed methods
on eight different real-world datasets of varying size.
With respect to optimization score
the greedy hill-climbing algorithm is found to perform better 
than the iterative ($k$-means-like) algorithm.
This is consistent for many different initialization strategies, 
but interestingly enough, 
the best performance is achieved when greedy is initialized with the solution found by the iterative.

We have also contrasted our methods with
\algrolx~\cite{henderson2012rolx}, 
a representative feature-based algorithm.
Direct comparison is not easy, 
as there is no available ground truth for the role-discovery problem, 
and as the \algrolx algorithm does not provide an optimization criterion.
Nevertheless, we find that, when measured with our objective function, 
\algrolx obtains high-cost solutions.
Compared as a neutral classification task,
where the aim is to predict the discovered roles by the corresponding features, 
the iterative algorithm achieves the highest accuracy.

The rest of this paper is organized as follows.
We present related work in Section~\ref{section:related}.
We discuss the preliminaries and define the problem in Section~\ref{sec:prel}.
In Section~\ref{section:perfect} we present algorithm for perfect role assignments,
and in Sections~\ref{section:greedy}--\ref{sec:kmeans} we present our discovery algorithms.
Section~\ref{sec:exp} is devoted for experimental evaluation, and we conclude the paper with discussion in Section~\ref{sec:conclusions}.




%\note{Decide whether we talk for ``nodes'' or ``vertices''.} I went with vertices



