\section{Preliminaries and\\problem definition}
\label{sec:prel}

We consider a graph $G = (V, E)$ 
with $|V|=n$ vertices and $|E|=m$ edges. 
Our goal is to assign roles to the vertices of the graph $G$.
We assume that the total number of roles $k$ is given.
A \emph{role assignment} 
$\funcdef{\role}{V}{[1, k]}$ is a function mapping each vertex $v$ to an integer between 1 and $k$, 
which is interpreted as a role id. 
Given a role assignment $\role$, 
the \emph{profile} of a vertex $v$ for that role assignment 
is a $k$-dimensional vector $\prof{v; \role} = \vect{x}$, 
where $x_i$, the $i$-th coordinate of $\prof{v; \role}$, 
is the number of vertices with role $i$ that are adjacent to $v$,
\[
	x_i = \abs{\set{(v, w) \in E \mid \role(w) = i}}.
\]
Our guiding principle for assigning roles to the graph vertices is that
\emph{vertices assigned to the same role should have more similar
profiles than vertices assigned to different roles.}

As vertex profiles are $k$-dimensional vectors, 
we quantify the similarity between them using the Euclidean distance 
$\dist{\vect{x},\vect{y}} = || \vect{x}-\vect{y} ||_2 = 
(\sum_{i=1}^k |x_i-y_i|^2)^{1/2}$, 
where $\vect{x}$ and $\vect{y}$ are vertex profiles.

Furthermore, each role $i\in[1,k]$ is represented by a $k$-dimensional vector $\centroid_i$, 
which is selected as a representative of the profile vectors of all vertices assigned to role $i$.
We use the Euclidean distance to define the distance between 
a vertex profile $\prof{v; \role}$ 
and a role representative vector $\centroid$.
For simplicity of notation we write 
\[
\dist{v, \centroid; \role} = 
\dist{\prof{v; \role},\centroid} = 
\norm{\prof{v; \role} - \centroid}_2.
\]
The role-mining problem we consider
can now be formulated as follows.

\begin{problem}
\label{problem:role-mining}
\emph{(\prbrm)}
Given a graph $G = (V, E)$ and an integer $k$, 
find a role assignment $\funcdef{\role}{V}{[1, k]}$ and 
$k$ representative role vectors $\centroid_1, \ldots, \centroid_k$
that minimize the cost
\[
\cost{\role, \centroid_1, \ldots, \centroid_k} = 
\sum_{v \in V} \dist{v, \centroid_{\role(v)}; \role}^2.
\]
\end{problem}

We can show that the \prbrm problem is  \np-hard.
The proof of the following proposition is obtained 
by a reduction from the \tmatch problem.
The proof is given in the appendix.

\begin{proposition}
\label{proposition:np}
\prbrm is \np-hard.
\end{proposition}

Intuitively, problem \prbrm resembles the $k$-means clustering problem.
However, a careful reader should immediately realize
that we are dealing with a much harder problem than $k$-means clustering.
To see this, notice that in the $k$-means problem 
we aim to cluster vectors whose coordinates are \emph{fixed}. 
In the \prbrm problem, however, we aim to cluster vertex profiles,
which are vectors whose coordinates depend on the role assignment.
Thus, we are working with a clustering problem
in which the data recursively depend on the output of the clustering problem itself.

To emphasize the difference between \prbrm and $k$-means clustering, 
consider the standard property of $k$-means algorithm:
\emph{for a fixed cluster membership it is easy to compute optimal representative vectors (centroids), 
and for fixed centroids it is easy to compute optimal cluster membership}. 

We can show that for the \prbrm problem
only the first part of the corresponding property holds.
Indeed, for a \emph{fixed} role assignment $\role$
the profiles of all vertices are also fixed.
The representative vector of a role
can then be easily computed as the centroid
of the profiles of all vertices that are assigned to that role. 

On the other hand, 
when the role centroids $\centroid_1, \ldots, \centroid_k$ are fixed
it is not easy to compute the optimal role assignment~$\role$.
We refer to this problem as \prbrmfixed.

\begin{problem} 
\label{problem:role-mining-fixed}
\emph{(\prbrmfixed)}
Assume a graph $G = (V, E)$ % an integer $k$,
and $k$ centroids $\centroid_1, \ldots, \centroid_k$.
Find a role assignment $\funcdef{\role}{V}{[1, k]}$ 
that minimizes the cost function
\[
\cost{\role} = 
\sum_{v \in V} \dist{v, \centroid_{\role(v)}; \role}^2.
\]
\end{problem}

The next proposition is proven in the appendix.

\begin{proposition}
\label{proposition:np-fixed}
Deciding whether an instance of \prbrmfixed has a zero-cost solution 
is an \np-complete problem. 
\end{proposition}

Not only does Proposition \ref{proposition:np-fixed} imply that \prbrmfixed is an \np-hard problem, 
but it also establishes that \prbrmfixed cannot be approximated to any multiplicative factor, 
no matter how large.

\begin{corollary}
Unless $\poly = \np$, there is no polynomial 
algorithm
that provides an approximation guarantee
to the \prbrmfixed problem.
\end{corollary}

Note that even though intuitively 
\prbrm is a more difficult problem than \prbrmfixed, 
the hardness result obtained for \prbrmfixed is much stronger
than the one obtained for \prbrm.
This could be just an artifact of our proof techniques
and it may be the case that \prbrm is also a hard problem to approximate. 
As we do not provide an approximation algorithm for \prbrm in this paper, 
its approximability is left as an open problem.



