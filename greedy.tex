\section{Hill-climbing algorithm}
\label{section:greedy}

The algorithm discussed in the previous section
returns a perfect (zero-cost) role assignment
but it does not put any constraint on the number of roles to be used.
In fact, as we will see in the experimental section, 
in most cases, \algperfect is forced to use a large number of roles. 
%This is an expected behavior, 
%as the requirement for a perfect role assignment is too rigid.

In this section, we return to the \prbrm problem
(defined in Problem~\ref{problem:role-mining})
and ask to find a minimum-cost role assignment
for a given number of roles $k$.
As the \prbrm problem is \np-hard (Proposition~\ref{proposition:np}) and
as a simple variant of the problem is \np-hard to approximate (Proposition~\ref{proposition:np-fixed}), 
we present a hill-climbing algorithm
that iteratively improves the cost of the role-assignment problem, 
until convergence.
The algorithm is presented and analyzed below.

Assume a role assignment $\role$ with optimal centroids $\centroid$. 
Let $v$ be a vertex, and let $j$ be an integer.
Define a new role assignment $\role'$ obtained from $\role$ by setting $\role'(v) = j$.
Let $\centroid'$ be the optimal centroids with respect to $\role'$.

We define the \emph{gain} to be the difference
of the value of the objective function for the two role assignments 
\[
\gain{v, j; \role} = 
	\sum_{v \in V} \dist{v, \centroid_{\role(v)}; \role} - \sum_{v \in V} \dist{v, \centroid'_{\role'(v)}; \role'}.
\]
A positive gain means that $\role'$ produces a smaller cost, making it a better role assignment.

\iffalse
i -> j
\[
	- c_i\norm{\centroid_i}^2 + (c_i - 1)\norm{\centroid_i'}^2
	+ c_j\norm{\centroid_j}^2 - (c_i + 1)\norm{\centroid_j'}^2
\]

\[
	+ \prof{w}_i^2  -  (\prof{w}_i - 1)^2
	- \prof{w}_j^2  +  (\prof{w}_j - 1)^2
\]

\[
	2\prof{w}_i - 2\prof{w}_j 
\]

\[
	-c_\ell ((\centroid_\ell)^2_i - (\centroid_\ell')^2_i
	-(\centroid_\ell)^2_j + (\centroid_\ell')^2_j)
\]

\[
	- 2(\centroid_\ell)_i  + 2(\centroid_\ell)_j  
\]
\fi

The proposed hill-climbing algorithm,
named \alggreedy, 
is illustrated as Algorithm~\ref{alg:greedy}. 
The algorithm starts with an initial role assignment ${\role}_{0}$.
Then it sequentially tries to improve the score by change the role of each vertex in the graph.
For each vertex $v$, it changes its assignment to the role $j$
that maximizes the gain $\gain{v, j}$. 
If there is no role that yields a positive gain, 
the role of $v$ remains unchanged.

\begin{algorithm}[t]
\caption{$\alggreedy(G, k, r_{0})$, hill-climbing algorithm.}
\label{alg:greedy}
initialize role assignment, $\role \define r_{0}$\; 
\While {changes} {
	\ForEach {$v \in V$} {
		$j^* \define \arg \max_j \gain{v, j}$\;
		\lIf {$\gain{v, j^*} > 0$} { reassign
			%update $\role$\;
			%update profiles and centroids\;
		}
	}
}
\end{algorithm}

For selecting the initial role assignment ${\role}_{0}$
we have experimented with a number of different strategies. 
More details are given in the experimental section.

\iffalse
As \alggreedy involves a number of nested loops, 
a na\"{i}ve implementation will make it prohibitively expensive for large networks.
However, we show that it is possible to implement \alggreedy 
in a much more efficient manner
so that the running time of the inner-most loop
({\bf foreach} loop in Algorithm~\ref{alg:greedy}) 
is linear in the size of the graph, 
and quadratic only in the number of roles $k$, 
which in practice can be assumed to be a very small constant.
\fi

\iffalse
\begin{proposition}
\label{prop:fastgain}
Assume a role assignment $\role$ with optimal centroids $\centroid$. 
Let $v$ be a vertex and $j$ be an integer.
Let $i = \role(v)$.
Define $n_\ell = \abs{\set{u \in V ; \role(u) = \ell}}$ to be the number of vertices having role $\ell$.
Define $d_\ell = \abs{\set{(v, w) \in E ; \role(w) = \ell}}$ to be the number of neighbors of $v$ having role $\ell$.

Define a new role assignment $\role'$ obtained from $\role$ by setting $\role'(v) = j$.
Let $n'_\ell = \abs{\set{u \in V ; \role'(u) = \ell}}$.
Let $\centroid'$ be the optimal centroids with respect to $\role'$. 
Let $\centroid^*$ be the ``intermediate'' centroids, where we have moved $v$ from $i$ to $j$ but we have not
updated the profiles of neighbors of $v$.  Then 
\[
\begin{split}
	\gain{v, j} = 
	& - n_i\norm{\centroid_i}^2 + n_i'\norm{\centroid_i^*}^2 - n_j\norm{\centroid_j}^2 + n_j'\norm{\centroid_j^*}^2 \\
	& + \sum_{(v, w) \in E} 2\prof{w, \role}_i - 2\prof{w, \role}_j  - 2  \\
	& - \sum_{\ell = 1}^k 2d_\ell \spr{(\centroid_\ell^*)_j  - (\centroid_\ell^*)_i - d_\ell / n_\ell' }.  \\
\end{split}
\]
\end{proposition}

\begin{proof}
A known identity $\sum (x_i - \centroid)^2 = \sum (x_i^2 - \centroid^2)$ allows us to 
rewrite the gain as
\[
\begin{split}
	\gain{v, j} = & \sum_{u \in V} \norm{\prof{u, \role}}^2 -  \norm{\prof{u, \role'}}^2 \\
	& - \sum_{\ell = 1}^k n_\ell \norm{\centroid_\ell}^2 -  n_\ell' \norm{\centroid_\ell'}^2.
\end{split}
\]
Let us define $B(w)$ to be a term in the first sum.
Let us abbreviate $\prof{w, \role}$ as $\prof{w}$.
Then $B(w) \neq 0$ if and only if $(v, w) \in E$,
and in such a case it is equal to
\[
\begin{split}
	B(w) & = \prof{w}_i^2 - (\prof{w}_i - 1)^2 + \prof{w}_j^2 - (\prof{w}_j + 1)^2 \\
	& = 2\prof{w}_i - 2\prof{w}_j  - 2.
\end{split}
\]
We now consider the second sum, which we will rewrite as
\[
	\sum_{\ell = 1}^k n_\ell \norm{\centroid_\ell}^2 -  n_\ell' \norm{\centroid_\ell^*}^2 +
	\sum_{\ell = 1}^k n_\ell' \norm{\centroid_\ell^*}^2 -  n_\ell' \norm{\centroid_\ell'}^2.
\]
The first sum, which we will denote by $A$ is equal to
\[
	A = n_i\norm{\centroid_i}^2 - n_i'\norm{\centroid_i^*}^2 + n_j\norm{\centroid_j}^2 - n_j'\norm{\centroid_j^*}^2. 
\]

Let us write $R(\ell)$ to be a term in the second sum. We can express this term as
\[
\begin{split}
	R(\ell) & = n_\ell'\spr{(\centroid_\ell^*)_i^2 - ((\centroid_\ell^*)_i - \gamma_\ell)^2 + (\centroid_\ell^*)_j^2 - ((\centroid_\ell^*)_j +  \gamma_\ell)^2} \\
	        & = 2d_\ell \spr{(\centroid_\ell^*)_i - (\centroid_\ell^*)_j -  \gamma_\ell}, \quad \text{where} \quad \gamma_\ell = d_\ell / n_\ell'.
\end{split}
\]
To conclude, we have
\[
	\gain{v, j} = - A + \sum_{(v, w) \in E} B(w) - \sum_{\ell = 1}^k R(\ell),
\]
which proves the identity.
\end{proof}
\fi



\begin{proposition}
The inner loop of \alggreedy
\emph{(}{\bf foreach} loop in Algorithm~\ref{alg:greedy}\emph{)} 
requires time $\bigO{k^2 n + km}$.
\label{prop:greedytime}
\end{proposition}

\iffalse
\begin{proof}
According to Proposition~\ref{prop:fastgain}, computing the gain can be done in time $\bigO{k + \dg{v}}$. 
Consequently, computing $j^*$ requires time $\bigO{k^2 + k\dg{v}}$. 
Updating the profiles, the centroids, and the role assignment $\role$
can be done in time $\bigO{k + \dg{v}}$. 
Summing over all the vertices gives us a total running time of $\bigO{k^2 n + km}$.
\end{proof}
\fi
